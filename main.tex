\documentclass[hidelinks, 12pt, a4paper,
  %oneside,      %% -- odkomentujte, pokud chcete svou práci mít pouze jednostrannou, mezera pro hřbet pak automaticky bude pouze na levé straně
 twoside,        %% -- pro oboustranné práce, mezera pro hřbet následně střídá strany.
 openright
]{report}

%% Nutné balíčky a nastavení
%%%%%%%%%%%%%%%%%%%%%%%%%%%%

\title{Systém pro vkládání a zpracování ankety/zpětné vazby} %% -- Název tvé práce
\author{Jan Hubený} %% -- tvé jméno
\date{2022} %% -- rok, kdy píšeš SOČku

\usepackage[top=2.5cm, bottom=2.5cm, left=3.5cm, right=1.5cm]{geometry} %% nastaví okraje, left -- vnitřní okraj, right -- vnější okraj
\usepackage{booktabs}

\usepackage{url} %% Balíček pro URL adresy
\def\UrlFont{} %% Nastavené "žádného" stylu u URL adres

\usepackage{pdfpages} %% Balíček umožňující vkládat stránky z PDF souborů
\usepackage{float} %% formátování obrázků
\usepackage[czech]{babel} %% balík babel pro sazbu v češtině
\usepackage[T1]{fontenc} %T1 kvůli češtině
\usepackage[utf8]{inputenc}
\usepackage{lmodern} %% vektorový font
\usepackage{hyphenat} %% dělení slov
\usepackage{cmap} %% balíček zajišťující, že vytvořené PDF bude prohledávatelné a kopírovatelné
\usepackage{graphicx} %% balík pro vkládání obrázků
\usepackage{subfig} %% balíček pro vkládání podobrázků
\usepackage{hyperref} %% balíček, který v PDF vytváří odkazy
\usepackage{etoolbox}

\patchcmd{\thebibliography}{*}{}{}{} %% přidání číslování pro nadpis zdrojů
\patchcmd{\listoffigures}{*}{}{}{} %% přidání číslování pro nadpis seznamu obrázků
\patchcmd{\listoftables}{*}{}{}{} %% přidání číslování pro nadpis seznamu tabulek

%% styly kapitol a sekcí
\usepackage[pagestyles,nobottomtitles]{titlesec} %% balíček pro úpravu stylu kapitol a sekcí
	
	%% definování nové sekce resp. jeho příkazu (5.)
	\titleclass{\subsubsubsection}{straight}[\subsection]
	\newcounter{subsubsubsection}[subsubsection]
	\renewcommand\thesubsubsubsection{\thesubsubsection.\arabic{subsubsubsection}}
	\makeatletter
	\def\toclevel@subsubsubsection{4}
	\def\l@subsubsubsection{\@dottedtocline{4}{11em}{4em}}
	\makeatother
	%%
	
	%% stylování jednotlivých nadpisů sekcí
	\titleformat{\chapter}[block]{\scshape\bfseries\huge}{\thechapter}{12pt}{\vspace{0pt}}[\vspace{-34pt}]
	\titleformat{\section}[block]{\scshape\bfseries\LARGE}{\thesection}{12pt}{\vspace{-10pt}}
	\titleformat{\subsection}[block]{\bfseries\Large}{\thesubsection}{12pt}{\vspace{-8pt}}
	\titleformat{\subsubsection}[block]{\bfseries\large}{\thesubsubsection}{12pt}{\vspace{-8pt}}
	
	\titleformat{\subsubsubsection}[block]{\normalfont\normalsize\bfseries}{\thesubsubsubsection}{12pt}{\vspace{-9pt}}
	\titlespacing*{\subsubsubsection}{0pt}{3.25ex plus 1ex minus .2ex}{1.5ex plus .2ex}
	%%
	
	%% vykreslování úrovní v nadpisech a v obsahu
	\setcounter{secnumdepth}{6}
	\setcounter{tocdepth}{3}
	%%
%% konec stylů kapitol a sekcí

\usepackage{fancyhdr}
\pagestyle{fancy}
\renewcommand{\headrulewidth}{1pt}

%% formátování odstavců
\setlength{\parindent}{0em}
\setlength{\parskip}{0,7em}

\linespread{1.15} %% řádkování

%% nastavení předcházení samotnému řádku textu na nové stránce
\widowpenalty10000
\clubpenalty10000

%% Začátek dokumentu
%%%%%%%%%%%%%%%%%%%%
\begin{document}

\pagestyle{empty}
\pagenumbering{gobble}
\begin{titlepage}
    \bfseries{
        \begin{center}
            \LARGE{STŘEDOŠKOLSKÁ ODBORNÁ ČINNOST}

            \vspace{14pt}
            \large{
                Obor č. 18: Informatika
            }

            \vspace{0.4 \textheight}

            \LARGE{
			Systém pro vkládání a zpracování ankety/zpětné vazby
            }

            \vspace{0.3\textheight}
        \end{center}
        
        \noindent\Large{Jan Hubený}

        \noindent\Large{Pardubický kraj \hspace{\stretch{1}}  Pardubice, 2022}
        
            
    }
\end{titlepage}

\cleardoublepage

%% Úvodní stránka s informacemi
{\bfseries
    \begin{center}
        \LARGE{STŘEDOŠKOLSKÁ ODBORNÁ ČINNOST}

        \vspace{14pt}
        {\large
            Obor č. 18: Informatika
        }

        \vspace{0.25 \textheight}

        \LARGE{
        	Systém pro vkládání a zpracování ankety/zpětné vazby
        }
    
    	\vspace{20pt}
    	
    	\LARGE{
    		Web application for processing surveys/feedback
    	}

        %%\LARGE{HansForms}

        \vspace{0.22\textheight}
    \end{center}  
}
{\Large
    \noindent\textbf{Autor:} Jan Hubený\\
    \textbf{Škola:} DELTA - Střední škola informatiky a ekonomie, s.r.o.\\Ke Kamenci 151, 530 03 Pardubice\\
    \textbf{Kraj:} Pardubický kraj\\
    \textbf{Konzultant:} RNDr. Jan Koupil, Ph.D.\\
}

\noindent Pardubice, 2022

\cleardoublepage

\noindent{\Large{\bfseries{Prohlášení}}}

\noindent Prohlašuji, že jsem svou práci SOČ vypracoval/a samostatně a použil/a jsem pouze prameny a literaturu uvedené v~seznamu bibliografických záznamů.

\noindent Nemám závažný důvod proti zpřístupňování této práce v~souladu se zákonem č. 121/2000 Sb., o~právu autorském, o~právech souvisejících s~právem autorským a o~změně některých zákonů (autorský zákon) ve znění pozdějších předpisů.  

\vspace{24 pt}

\noindent V~Pardubicích dne 25. Března 2022 \dotfill{}\hspace{\stretch{0.4}} 

\hspace{8.5cm} Jan Hubený

\cleardoublepage

\vspace*{0.8\textheight}
\noindent{\Large{\bfseries{Poděkování}}}

\noindent
Prvně bych chtěl poděkovat mému vedoucímu práce RNDr. Janu Koupilovi, Ph.D. za odborné vedení projektu a věcné připomínky. Dále děkuji Bc. Tomáši Vanišovi za pomoc při prvotních krocích při tvorbě aplikace a za pomoc při výběru technologií pro tento projekt. V~neposlední řadě také děkuji své rodině, přátelům a spolužákům za psychickou podporu, za účast na testování aplikace a za nápady pro zlepšení aplikace.

\cleardoublepage

\noindent{\Large{\bfseries{Anotace}}}

\noindent 
Práce dokumentuje vývoj a používání webové aplikace pro zpracování ankety nebo zpětné vazby. Systém je založen na backendovém frameworku Laravel, frontendovém frameworku VueJS a na databázovém systému PostgreSQL. Uživatele v~aplikaci mohou jednak formuláře vyplňovat, ale i spravovat. Kromě základní funkcionality jako je vykreslení formuláře a následná možnost odpovědi aplikace obsahuje např. omezení přístupu datem spuštění a ukončení, privátní přístup pozvaným uživatelům či export dat pro účely korelační analýzy.

\vspace{18pt}

\noindent{\Large{\bfseries{Klíčová slova}}}

\noindent Webová aplikace; Zpracování ankety; Laravel, VueJS, PostgreSQL

\vspace{18pt}

\noindent{\Large{\bfseries{Annotation}}}

\noindent
The thesis documents the development and use of a web application for processing surveys or feedback. The system is based on the backend framework Laravel, the frontend framework VueJS and the database system PostgreSQL. Users can both fill out and manage forms in the application. In addition to basic functionality such as form rendering and subsequent response option, the application includes for example access restriction by start and end date, private access for invited users or data export for correlation analysis.

\vspace{18pt}

\noindent{\Large{\bfseries{Keywords}}}

\noindent Web application; Processing surveys; Laravel, VueJS, PostgreSQL

\cleardoublepage

\tableofcontents

\cleardoublepage

\pagenumbering{arabic}
\pagestyle{plain}
\setcounter{page}{15} %% úvod by neměla být první stránka, začína na čísle 'součet předchozích + 1'

\chapter{Úvod}

Anketa a sběr dat zpětné vždy vazby byly, jsou a budou potřeba pro různé druhy lidských aktivit jak v reálném světe, tak i na internetu. V dnešní době se lidé mohou zajímat např. o výsledky ankety zhodnocující průběh pobytu v zahraničí či zpětnou vazbu k nějaké veřejné akci. Tato data nám mohou pomoci např. při zlepšování služeb nebo pro vyhodnocení úspěšnosti a popularity dané akce či produktu. A právě pro tyto účely jsou tvořeny webové anketní systémy, které nám zajištují plnou kontrolu nad zmíněnými daty, což má být i výstupem této práce.

Cílem práce je vytvořit intuitivní, jednoduchý, a přitom komplexní webový systém pro správu a zpracování ankety – tzn. možnost vytvoření formuláře, dále možnou úpravu existujícího formuláře a následné publikování formuláře pro zisk dat zpětné vazby.

\chapter{Existující dostupná řešení}

Webových systému pro zpracování ankety je na trhu opravdu nepřeberné množství. Některé z nich bych zde rád představil, jelikož byly inspirací pro mou vlastní implementaci.

\section{Google Forms}

Google Forms (česky Google Formuláře) je webová aplikace určená k tvorbě online průzkumů a kvízů. Je součástí balíčku webových aplikací od Googlu, ve kterých nalezneme např. Google Sheets či Google Docs. Při tvorbě těchto formulářů máme možnost pracovat s různými předpřipravenými šablonami nebo s předdefinovanými typy otázek. Do formulářů můžeme komponovat i různá multimédia jako např. videa či obrázky a samozřejmě zde nechybí i vizualizace výsledků. Jediné, co potřebujeme k tvorbě formulářů na této platformě, je vytvoření a přihlášení do Google účtu. \cite{GoogleForms1}\cite{GoogleForms2}

\section{Microsoft Forms}
Microsoft Forms (česky Microsoft Formuláře) je služba umožňující tvorbu a sdílení dotazníků a kvízů určený jak pro soukromé účely, tak pro firemní sféru. Je součástí balíčku aplikací Microsoft Office, kam patří např. Microsoft Word nebo Microsoft Excel. I zde máme při tvorbě ankety možnost vybrat z předdefinovaných typů otázek společně s např. možností nahrání souboru či zobrazení výsledků. K tvorbě ankety se stačí jen přihlásit do svého Microsoft účtu. \cite{MSForms1} \cite{MSForms2}

\section{Survio}
Survio je anketní nástroj pro měření zákaznické spokojenosti, marketingový průzkum či jiné účely původem z Česka. Kromě samotného tvoření formuláře nabízí i funkce jak např. export dotazníku do PDF, souhrnnou statistiku ve grafech a tabulkách či zpracování výsledků v reálném čase. Tvoření ankety se základními funkcemi je zdarma, služba ale nabízí i možnost prémiových funkcí. K tvorbě formuláře je znovu třeba se přihlásit. \cite{Survio}


\chapter{Použité technologie}
Tato kapitola shrnuje a představuje hlavní technologie využité při řešení a implementaci projektu.

\section{Frontend}
Za frontend je v~souvislosti s~programováním a návrhem aplikace považováno to, co běžně vidí uživatel dané aplikace. Jedná se o~vizuální stránku programu, díky které lze komunikovat s~tzv. backendem (viz dále) – to znamená, že musí jednak správně interagovat s~uživatelem (např. umožnit uživateli si v~e-shopu vybrat zboží, zpracovat další potřebné údaje poskytnuté uživatelem a následně zboží objednat), ale i se serverem (např. posílat validní data a instrukce od uživatele na server).

Lze ji také označit za „klientskou stranu“ aplikace, jelikož je tvořena primárně pro běžnou a intuitivní obsluhu lidmi, kteří nemusí a ani nepotřebují rozumět technologiím, na kterých tato část funguje. \cite{FEvsBE}

	\subsection{JavaScript}
	JavaScript je multiplatformní, objektově orientovaný, událostmi řízený skriptovací jazyk, který je používán primárně při tvorbě webových stránek. Je určen především pro tvorbu klientské strany, ale např. díky prostředí Node.js lze v~něm dnes psát i serverovou část aplikace. \cite{JS1}\cite{JS2}
	
	Psát web v~čistém JavaScriptu dnes v~souvislosti s~pokročilejšími webovými aplikacemi s~největší pravděpodobností postrádá smysl, jelikož zde existují tzv. javascriptové frameworky, které za nás řeší nejrůznější problémy (jak technické, tak i bezpečnostní), které bychom museli implementovat v~čistém JavaScriptu ručně. Tento projekt není výjimkou, a proto byl při tvoření klientské strany použit framework VueJS, který je představen v~sekci \ref{sec:vuejs}.
	
	\subsection{VueJS}\label{sec:vuejs}
	VueJS je progresivní open-source framework určený k~tvorbě uživatelských rozhraní. Jedná se o~framework psaný v~jazyce JavaScript, ale lze ho psát i v~Typescriptu. Tento framework je vhodný primárně jen pro účely viditelné části aplikace – nelze v~něm např. programovat chování serveru. \cite{VueJS1}
	
	Svou popularitu si získal především tím, že je vyvíjen komunitou, a strukturou kódu, ve kterém je psán (viz obrázek \ref{fig:vue_kod_komponenty}). Kód jednotlivých komponent je tvořen třemi základními částmi:
	
	\begin{itemize}
		\item \textbf{Template} - část určená pro návrh struktury komponenty v~HTML
		\item \textbf{Script} - část určená pro tvoření logiky a manipulaci s~daty v~komponentě v~JavaScriptu
		\item \textbf{Style} - část určená pro stylování komponenty v~CSS
	\end{itemize}

	Nutno zmínit, že se jedná o~framework, který je vytvořen právě pro práci s~komponentami – znamená to, že je vhodné jednotlivé části aplikace rozdělovat na jednotlivé komponenty, které lze použít později v~jiném místě aplikace. Tento přístup vede k~udržitelnosti aplikace. \cite{VueJSSyntax}\cite{VueJS2}
	
	\begin{figure}[h]
		\centering
		\includegraphics[width=0.8\textwidth]{img/vue_kod_komponenty.png}
		\caption{Ukázka struktury kódu VueJS komponenty}
		\label{fig:vue_kod_komponenty}
	\end{figure}
	
	\subsection{Bootstrap}
	Bootstrap je open-source framework určený především pro stylování běžných komponent webových aplikací. Styly se aplikují na HTML ve formě tříd, kterým jsou interně předepsány parametry, jakým způsobem má být daný element nastylován. Také obsahuje předem hotové styly pro běžné komponenty jako jsou např. formuláře a jejich prvky, navigační lišty, modální okna atp. \cite{Bootstrap1}\cite{Bootstrap2}\cite{Bootstrap3}
	
	\subsection{NPM}\label{sec:npm}
	NPM je rozsáhlý open-source správce balíčků pro JavaScript. Je využíván zejména k~přidávání komunitou vytvořených balíčků do vlastních projektů či k~sdílení vlastních balíčků - ty jsou uloženy v~tzv. registru, což je veřejná databáze javascriptového softwaru. K~instalaci balíčků se nejčastěji používá CLI (Command Line Interface), kterým správce NPM disponuje. \cite{NPM}
	
	V~projektu byl využit zejména k~přidání a správě balíčků na frontendové části společně s~VueJS.

\section{Backend}
Termínem backend označujeme část aplikace, která pro běžného uživatele zpravidla není vidět. Jedná se o~administrátorskou část, ve které administrátor konfiguruje, jak se aplikace bude při jednotlivých situacích chovat. Backend, jako samostatná část aplikace, poté slouží ke zpracování dat – např. vyřizuje příchozí požadavky od frontendu nebo na frontend sama zasílá konkrétní data.

Backend může být také zamýšlen jako prostředí pro správu např. obsahu webu. Příkladem může být e-shop, kde na backendu bude moci administrátor např. přidávat jednotlivé zboží nebo spravovat veškeré objednávky.  \cite{BE1}\cite{BE2}
	
	\subsection{PHP}
	PHP (zkr. pro Hypertext Preprocessor) je open-source multiplatformní skriptovací jazyk. Je používán především pro vývoj webových aplikací a může být vložen přímo v~souborech HTML. V~rámci základních informací lze zmínit, že podporuje procedurální i objektově-orientované programování a dokáže pracovat s~velkým množstvím databázových systémů. Oproti JavaScriptu je kód spouštěn přímo na serveru, kde generuje HTML, které je předáváno dále na klientskou stranu uživateli. \cite{PHP1}\cite{PHP2}
	
	Hlavním důvodem, proč byl pro tento projekt jazyk PHP vybrán, byla jednoduchost syntaxe jazyka a obecně nižší provozní náklady. V~jazyce PHP je napsán framework Laravel, ve kterém je tvořena backendová část projektu.
	
	\subsection{Laravel}
	Laravel je open-source framework pro tvorbu webových aplikací v~jazyce PHP. Řeší za vývojáře mnoho běžně implementovaných funkcí jako např. autentifikaci, přesměrovávání, správu sessions a cookies souborů, tvoření databázových migrací nebo testování aplikace. Je inspirován frameworky jako Ruby on Rails, ASP.NET MVC nebo Sinatra a snaží se o~to, aby vývoj webových aplikací byl pro vývojáře co nejpohodlnější a nejintuitivnější. \cite{Laravel1} Framework vychází z~návrhového vzoru MVC \cite{LaravelMVC}.
	
	Dále samotný Laravel disponuje velkým ekosystémem přídavků, funkcí a knihoven, díky kterým si můžeme při vývoji ještě více usnadnit práci. \cite{LaravelEco} V~mém projektu ze zmíněného ekosystému využívám např. Laravel Sanctum pro autentifikaci single-page aplikací.
	
		\subsubsection{Návrhový vzor MVC}
		MVC (zkr. pro Model-View-Controller) je architektonický vzor, jehož základní myšlenkou je oddělení logiky od výstupu. Obsahuje (jak už z~názvu vyplývá) tři základní komponenty:
		
		\begin{itemize}
			\item \textbf{Model} – obsahuje veškerou logiku spojenou s~daty (např. databázové dotazy); neřeší, odkud data přišla a ani jak budou dále předána uživateli
			\item \textbf{View} (v~překladu Pohled) – řeší, jak budou data vyobrazena uživateli; neřeší, odkud mu data přišla
			\item \textbf{Controller} (v~překladu Řadič nebo Kontrolér) – tzv. prostředník mezi modely a pohledy – propojuje je; reaguje na podněty uživatele
		\end{itemize}
	
		Jeho účelem je zamezení tvorby tzv. „špagetového kódu“ – tj. dlouhý soubor, který obsahuje implementaci výše zmíněných funkcionalit (vše na jednom místě), což je v~konečném důsledku velmi nepřehledné. Také nám usnadňuje myšlení při vývoji projektu – jednotlivé části kódu se umisťují tam, kam patří (např. stylování a konečný výstup pro uživatele do pohledu). \cite{MVC} Pro lepší představu je zde přiložena grafická podoba tohoto konceptu (resp. diagram), který je znázorněn v~obrázku \ref{fig:mvc_diagram}.
		
		\begin{figure}[h]
			\centering
			\includegraphics[width=0.6\textwidth]{img/mvc_diagram.png}
			\caption{MVC diagram \cite{MVCDiagram}}
			\label{fig:mvc_diagram}
		\end{figure}
		
	\subsection{Composer}\label{sec:composer}
	Composer je nástroj pro správu závislostí pro jazyk PHP. Umožňuje nám v~projektu deklarovat, které cizí knihovny (resp. balíčky) budou v~projektu použity. Ty poté dokáže nainstalovat nebo aktualizovat. Funkcionalitou je velmi podobný jako NPM (viz sekce~\ref{sec:npm}). \cite{Composer}
	
	V~tomto projektu byl použit primárně ke správě balíčků pro backendovou část.
		
\section{Úložiště dat}
Samotná data, která jsou spravována a vyměňována mezi backendem a frontendem, musí být samozřejmě někde ukládána. Kromě zmíněných dat samozřejmě musí být i samotná aplikace někde uložena a veřejně dostupná. K~těmto účelům slouží právě úložiště dat – o~využitých úložištích v~tomto projektu je psáno níže.

	\subsection{Webhosting}
	Webhosting je služba, díky které si můžeme pronajmout prostor na cizím vzdáleném serveru pro naše webové stránky. Poskytovatelé těchto služeb většinou nabízí i služby spojené s~provozem databází nebo e-mailové schránky. Hlavní výhodou této služby je fakt, že nemusíme fyzicky vlastnit žádný server a že nemusíme řešit administrativní úkony s~ním spojené (např. instalace specializovaného softwaru na serveru). \cite{Webhosting}
	
	Pro účely tohoto projektu bylo potřeba mít jak webový server pro nahrání celé aplikace, tak i databázový server pro uchovávání dat. O~využitých poskytovatelích těchto služeb je psáno v~práci dále. 
	
		\subsubsection{DigitalOcean}\label{sec:do}
		DigitalOcean je americký poskytovatel cloudových služeb s~datovými centry po celém světě. Nabízí obrovské množství technologií a služeb – např. správu webového serveru, virtuální stroje či možnost zprostředkování databázového serveru. Celkově se společnost snaží především o~poskytování služeb, které jsou uživatelsky přívětivé. \cite{DO1}\cite{DO2}
		
		Z~balíčku služeb, které DigitalOcean nabízí, byla pro tento projekt použita takzvaná \uv{App Platform} \cite{DO3}, do které lze jednoduše nahrát soubory webové aplikace a následně využívat webové služby.
		
		\subsubsection{Heroku}\label{sec:heroku}
		Heroku je cloudová platforma, která je využívána k~nasazení, správě a škálování moderních aplikací. V~rámci nabízených služeb se velmi podobá DigitalOceanu – nabízí tedy např. možnost využití virtuálních strojů a služby s~tím spojené nebo možnost zprostředkování nejrůznějších databázových serverů. \cite{Heroku1}\cite{Heroku2}
		
		V~rámci této implementace byla od Heroku využita služba zprostředkovávající databázové služby – konkrétně PostgreSQL server. 
		
	\subsection{Databáze}
	Databáze je organizovaná kolekce strukturovaných dat, která je v~dnešní době běžně uchovávána elektronicky. Je většinou řízena tzv. database management systémem (DBMS), který zajištuje veškerou administrativu nad daty. Data a DBMS dohromady tvoří databázový systém. Cílem moderní databáze je např. zajistit zpracování a uchovávání velkého množství dat, zabezpečení dat, dostupnost dat, možnost správy a údržby systému či škálovatelnost. Za předchůdce databáze lze považovat např. kartotéku u~lékaře. \cite{DBSummary}
	
	
	
	Databáze můžeme dělit podle modelu, který svou strukturou a funkcionalitou splňují. Základní modely jsou uvedeny níže: \cite{DBModel}
		
	\begin{itemize}
		\item \textbf{Hierarchický model} – data jsou uspořádána do stromové struktury, kde jsou omezena vztahem „jeden k~více“ (záznam z~vyšší úrovně lze spojit jen se záznamy nižší úrovně), jsou identifikovány tzv. ukazatelem (ten ukazuje na místo v~databázi, kde jsou data uložena). \cite{HierarchDB}
		\item \textbf{Síťový model} – data jsou uspořádána jako uzly rovinného grafu, jsou identifikovány tzv. ukazatelem; oproti hierarchickému modelu lze spojovat záznamy libovolně. \cite{SitDB} 
		\item \textbf{Relační model} – data jsou uspořádána v~tabulkách do relací (tj. spojení dat, která obsahují nějaký společný atribut), jsou identifikovány primárním klíčem. \cite{RelacDB}
		\item \textbf{Objektově-orientovaný model} – data jsou reprezentována jako objekty, které vychází z~objektově-orientovaného programování. \cite{OOPDB}
	\end{itemize}
	
	K~zajištění možnosti ovládání databáze musí být také zajištěn způsob, jakým lze s~databází komunikovat. K~těmto účelům vznikly tzv. dotazovací jazyky, které právě slouží ke komunikaci uživatele s~příslušným programem \cite{DotazJazyk} - zde databázovým systémem. Mezi nejpoužívanější jazyky tohoto druhu patří jazyk SQL, který je primárně používán s~relačními databázemi \cite{SQL}.
	
	Dalším důležitým faktorem je spolehlivost databáze. K~tomuto účelu slouží tzv. zásady ACID - ty garantují, že data po jednotlivých úkonech nebo transakcích (zjednodušeně skupiny úkonů \cite{Transakce}) - např. aktualizace dat - budou přesná a konzistentní i při selhání systému. Zkratku ACID lze rozdělit na tyto části:
	
	\begin{itemize}
		\item \textbf{Atomicity} (v~překladu Atomicita) – buď se transakce provede celá, nebo se data nijak nezmění \cite{ACID}
		\item \textbf{Consistency} (v~překladu Konzistence) – data musí být validní a konzistentní - tzn. že musí splňovat požadavky databázového systému např. platný typ vstupní hodnoty pro určitý atribut \cite{ACID}
		\item \textbf{Isolation} (v~překladu Izolace) – všechny transakce musí být vzájemně nezávislé a nesmí se nějak ovlivňovat \cite{ACID}
		\item \textbf{Durability} (v~překladu Odolnost) – po provedení transakce musí být data permanentně uložena v~databázi \cite{ACID}
	\end{itemize}
	
	V~projektu byla využita relační databáze PostgreSQL, která je popsána v~sekci \ref{sec:pgsql}.
	
		\subsubsection{PostgreSQL}\label{sec:pgsql}
		PostgreSQL je open-source objektově-relační databáze (spojení relačního a objektově orientovaného modelu) splňující ACID zásady. Je rozšířená a oblíbená kvůli její architektuře, výkonnosti, spolehlivosti, datové integritě, rozšířitelnosti, škálovatelnosti a velkému množství funkcí. Využívá jazyk SQL, který rozšiřuje o~další funkce a syntaxi. \cite{PostgreSQL}

\chapter{Vlastní implementace}

\section{Úchovávání dat}

Jak již bylo zmíněno, uživatelská data a samotná aplikace musí být někde uchovávány. Webová aplikace (tj. VueJS frontend a Laravel backend) je nasazena pomocí služby "App Platform" od poskytovatele DigitalOcean. PostgreSQL databáze pro administraci nad uživatelskými daty a formuláři, jejíž struktura bude rozebrána v této části, je nasazena na Heroku serveru. 

\subsection{Struktura databáze}
K tomu, aby byla struktura databáze vysvětlena, je níže vygenerováno schéma, které by mělo sloužit pro lepší vizualizaci.

\begin{figure}[h]
	\centering %% příkaz, který ti obrázek zarovná na střed
	\includegraphics[width=0.6\textwidth]{img/db_diagram.png} %% vložení samotného obrátku
	\caption{Diagram struktury databáze vygenerovaný aplikací DataGrip !!!citace!!!} %% popisek obrázku, nezapomeň na citace!
	\label{fig:db_diagram} %% označení až budeš chtít na obrázek odkazovat
\end{figure}

Jak je již z obrázku vidět, databáze je tvořena tabulkami, které jsou provázány určitými relacemi. Všechny tabulky mají unikátní primární klíč (ve všech tabulkách kromě \textit{password\_resets} atribut \textit{id}), kterým lze identifikovat jednotlivé instance. Většina z nich má také tzv. cizí klíč, který slouží k propojení s jinou tabulkou na základě shody tohoto klíče s cizím primárním klíčem. Také většina obsahuje předgenerované atributy timestamps (v překladu časová razítka) \textit{created\_at} a \textit{updated\_at}, které nám umožňují sledovat, kdy byla položka vytvořena či upravena.

Základním objektem databáze je samotný uživatel (tabulka \textit{users}), který obsahuje standardní vlastnosti jako jsou: uživatelské jméno (\textit{name}), emailovou adresu (\textit{email}), která také slouží jako identifikátor (musí být unikátní), heslo v zašifrované podobě (\textit{password}) a další předgenerované atributy, které jsou využívány primárně samotným frameworkem Laravel.

Dalším důležitým základním objektem je formulář (tabulka \textit{forms}). Tento objekt reprezentuje samotný dotazník a jeho vlastnosti.

%databázové objekty a jejich vlastnosti
\section{Backend}
	V rámci backendové nebo-li serverové části projektu je prvně nutné zmínit, že funguje na principu API - tzn. že na ní přicházejí požadavky od klienta a na ně náležitě odpovídá. Nevrací ani nijak nespravuje vizualizaci dat - jen předává data na frontendovou část, o které je psáno dále.
	
	Zde jsou zobecněné základní úkony, které v mé implementaci právě backend vykonává:
	
	\begin{itemize}
		\item Zpracování a vyřizování požadavků z webové frontendové části
		\item Administraci databáze - tzn. vytváření, úpravu, mazání a čtení jednotlivých objektů a migrace databáze (tj. automatizovaná deklarace struktury databázového objektu)
		\item Rozesílání emailů (např. pozvánky na neveřejný formulář)
		\item Exportování dat do Excel tabulek
		\item Autentifikaci uživatele
	\end{itemize}
	 
	V následující části jsou popsány jednotlivé backendové komponenty a principy, díky kterým je celý projekt implementován.
	
	\subsection{Obecná struktura Laravel projektu}
	Prvně je nutné rozebrat obecnou strukturu Laravel projektu. 
		\begin{figure}[H]
			\centering %% příkaz, který ti obrázek zarovná na střed
			\includegraphics[width=0.9\textwidth]{img/laravel_struktura.png} %% vložení samotného obrátku
			\caption{Obecná struktura nově vygenerovaného Laravel projektu} %% popisek obrázku, nezapomeň na citace!
			\label{fig:laravel_str} %% označení až budeš chtít na obrázek odkazovat
		\end{figure}
	%% obrázek struktury složky
	Jak můžeme na obrázku výše vidět, celý projekt je složen z mnoha složek a souborů. Jelikož podrobný rozbor jednotlivých částí není předmětem této práce, tak jsou nejdůležitější části pro implementaci projektu popsány níže jen obecně.
	\begin{itemize}
		\item Složka \textit{app} obsahuje většinu tříd jádra aplikace
		\item Složka \textit{bootstrap} obsahuje soubory pro zavedení a spuštění aplikace
		\item Složka \textit{config} obsahuje konfigurace jednotlivých částí aplikace
		\item Složka \textit{database} obsahuje soubory spojené s prací s databázi
		\item Složka \textit{lang} obsahuje soubory jazyků a překladů (zde nevyužita)
		\item Složka \textit{public} obsahuje soubory, které jsou veřejně dostupné při uživatelské interakci s aplikací (např. při načtení webu v prohlížeči)
		\item Složka \textit{resources} obsahuje pohledy a nezkompilované soubory pro celý frontend
		\item Složka \textit{routes} obsahuje všechny definice cest aplikace (např. cestu \textit{/api/forms/create} pro vyřízení požadavku vytvoření nového formuláře)
		\item Složka \textit{storage} obsahuje primárně záznamy o chodu aplikace a jiné aplikací vygenerované soubory
		\item Složka \textit{tests} obsahuje automatizované testy aplikace (zde nevyužita)
		\item Složka \textit{vendor} obsahuje závislosti a soubory přídavných balíčků, které aplikace používá 
		\item Soubor .env držící veškeré důležité administrativní hodnoty jako např. přihlašovací údaje k databázi
		\item Soubor artisan obsahující důležité příkazy pro stavbu aplikace \cite{LaravelArtisan}
		\item Soubor composer.json držící informace o přídavných balíčcích pro Laravel projekt \cite{ComposerpJSON}
		\item Soubor package.json držící informace o přídavných balíčcích pro frontendovou část projektu \cite{NPMpJSON}
		\item Soubor webpack.mix.js obsahující informace pro kompilaci souborů pro frontend \cite{LaravelJSCSS}
	\end{itemize} \cite{LaravelDir}

	\subsection{Modely}
	Modely slouží k mapování jednotlivých dat z databáze a jsou zprostředkovány pomocí objektově-relačního mapovacího balíčku Eloquent. Pro správnou funkčnost musí být zajištěno, že má každá tabulka v databázi vlastní model. Díky tomuto přístupu lze s daty manipulovat tak, že vytvoříme instanci příslušné třídy a na ní voláme příslušné metody (např. \textit{update} pro změnu záznamů). *cite https://laravel.com/docs/8.x/eloquent*
	
	V rámci tohoto projektu vzniklo mnoho modelů, které mapují většinu tabulek databáze (v kontextu mnou vytvořených modelů nejsou vytvořeny modely např. pro předgenerované tabulky jako \textit{migrations} nebo \textit{failed\_jobs}). Kromě samotné funkce mapování dat z databáze jim můžeme přiřazovat různé vlastnosti a metody, díky kterým lze měnit např. chování při předávání dat. Nejdůležitější a mnou použité jsou v následujícím seznamu:
	
	\begin{itemize}
		\item Metody vytvářející vazby mezi modely - Ty určují jednotlivé vztahy mezi modely a slouží k jednodušší práci s daty. Existuje mnoho metod - např. belongsTo (označuje, komu samotný model náleží) či hasOne/hasMany (označuje, které model/y samotnému modelu patří).
		\item Vlastnost \textit{fillable} - Ta určuje, do kterých atributů může být na vrstvě webové aplikace zapisováno. Nenapíšeme sem tedy např. \textit{id}, které většinou chceme doplnit až při uložení do databáze samotnou databází.
		\item Vlastnost visible - Ta určuje, které atributy jsou ve vrácených datech z databáze viditelné a tedy poslané dále (myšleno na frontend aplikace). Např. z bezpečnostního hlediska sem nenapíšeme atribut uživatelského hesla, který by neměl být obecně předáván mimo server.
		\item Vlastnost table - Ta slouží k přesnému určení tabulky pomocí jejího jména, ke které se vytvořený model vztahuje.
		\item Vlastnost with - Ta slouží k výchozímu připojení dalších dat z jiného modelu ke stávajícímu setu dat modelu. K takovému úkonu je nutné mít vytvořené vazby pomocí příslušných metod.
	\end{itemize} *cite https://laravel.com/docs/8.x/eloquent*

	Každý model většinou nevyužívá všechny zmíněné činnosti, ale jen ty, které jsou pro jeho typ důležité. Příkladem je zde uveden poměrně rozsáhlý model formuláře \textit{Form}. Podobným způsobem jsou vytvořeny i ostatní modely.
	
	Třída modelu Form reprezentuje jednotlivé formuláře. Odkazuje se na stejnojmennou tabulku \textit{forms}, ze které přebírá data. Má definované vlastnosti: \textit{fillable} (např. jméno formuláře, popis formuláře, čas spuštění, ID uživatele, kterému formulář patří,...), \textit{visible} (např. atributy jako jméno formuláře, odkaz na formulář či popis formuláře, ale také názvy atributů, které jsou přidávány v průběhu práce s daty, a názvy metod vytvářejících vazby mezi některými dalšími modely jako např. pro model uživatele, pro kterého platí, že mu formulář musí náležet) a \textit{with} (zde jen název vazbové metody pro model elementu formuláře, u kterého platí, že musí formuláři náležet a že jednomu formuláři může náležet i více elementů). Samotná definice provázání modelu s tabulkou v databázi pomocí vlastnosti table zde není potřeba - Laravel umí u jednoduše pojmenovaných modelů vygenerovaných pomocí Artisan skriptů tuto vlastnost vnitřně doplnit.
	
	%%obrázek modelu php
	\begin{figure}[H]
		\begin{figure}[H]
			\centering
			\includegraphics[width=0.8\textwidth]{img/form_model/vlastnosti.png}
			\caption{Zmíněné vlastnosti modelu formuláře}
			\label{fig:model_vlastnosti}
		\end{figure}
		
		\begin{figure}[H]
			\centering
			\includegraphics[width=0.7\textwidth]{img/form_model/metody.png}
			\caption{Zmíněné metody modelu formuláře}
			\label{fig:model_metody}
		\end{figure}
	\end{figure}
	
	
	
			
	\subsection{Kontrolery}
	
	
	\subsection{Exporty}
	
	\subsection{Maily}
	
	\subsection{Migrace databáze}
		\subsubsection{Factories}
	
	\subsection{Pohledy}
	
	\subsection{Routes}
	
	\subsection{Průběh jednotlivých činností}
		\subsubsection{Ukládání formuláře}
		\subsubsection{Mazání formuláře}
		%% atd.



\section{Frontend}\label{sec:impl_frontend}
Frontendová část tohoto projektu zajišťuje především komunikaci mezi uživatelem a serverem pomocí prostředí, díky kterému je tato komunikace snazší a obecně přívětivější. Serverová část očekává příchozí požadavky od této části na cesty, které byly už definovány (viz sekce \ref{sec:routes}).

V rámci této implementace (a z podstaty využitých technologií) probíhá komunikace s API (viz sekce \ref{sec:impl_backend}), které se tato část snaží poskytovat validní data. Slovo \uv{snaží} je zmíněno proto, jelikož se na ni v reálném světe nemůžeme úplně spoléhat - frontendovou část může uživatel s dostatečnými znalostmi a zkušenostmi modifikovat, čímž může i změnit způsob komunikace se serverem. To ale neznamená, že na ní nemůže probíhat jakákoliv validace a kontrola dat - frontend pro správnou funkčnost celé aplikace musí serveru ve výchozím (uživatelem nezměněném) stavu backendu poskytovat validní data - backend je ale musí preventivně validovat taky.

Uživatelské prostředí musí být připraveno tak, aby bylo příjemné k užívání a ergonomické. Zároveň by nemělo být nějak složité a přehlcené ovládacími prvky, což by mohlo uživatele od používání aplikace odradit. Dále by toto prostředí mělo uživatele \uv{vést} úkonem, který chce uživatel provést - tzn. pokud zadává neplatná data, tak ho upozornit a rozhodně počkat na opravu dat před odesláním požadavku na server. Uživatelské prostředí, které v rámci tohoto projektu vzniklo, je určeno primárně pro vykreslování ve webovém prohlížeči.

V následující části jsou popsány jednotlivé frontendové komponenty a principy, díky kterým může celá aplikace fungovat.
	
	\subsection{Obecná struktura VueJS aplikace}\label{sec:obecna_str_vuejs}
	Nejprve je nutné rozebrat obecnou strukturu VueJS projektu.
	
	\begin{figure}[H]
		\centering
		\includegraphics[width=0.6\textwidth]{img/vuejs_struktura.png} 
		\caption{Obecná struktura nově vygenerovaného VueJS projektu}
		\label{fig:vuejs_str}
	\end{figure}

	Jak můžeme na obrázku \ref{fig:vuejs_str} vidět, celý projekt je složen z mnoha složek a souborů. Podrobný rozbor všech souborů a složek není předmětem této práce, proto jsou nejdůležitější zmíněné části popsány níže jen obecně.
	
	\begin{itemize}
		\item Složka \textit{dist} obsahuje zkompilovaný kód aplikace
		\item Složka \textit{node\_modules} obsahuje závislosti a soubory přídavných balíčků NPM (viz sekce \ref{sec:npm}), které aplikace používá
		\item Složka \textit{public} obsahuje obecně veřejné soubory (např. favicon)
		\item Složka \textit{src} obsahuje všechny soubory se zdrojovým kódem, ze kterých se poté tvoří zkompilovaný kód celé aplikace
		\item Soubor \textit{package.json} obsahuje informace o přídavných balíčcích spravovaných pomocí NPM (viz sekce \ref{sec:npm})
	\end{itemize}

	*cite https://5balloons.info/project-tour-of-vue-cli-app/*
	*cite https://stackoverflow.com/questions/22842691/what-is-the-meaning-of-the-dist-directory-in-open-source-projects*
	
	Popsaná struktura výše je standardní pro oddělenou frontendovou aplikaci. To v implementaci tohoto projektu ale neplatí, jelikož je frontend přímo zaintegrován v Laravel projektu (viz sekce \ref{sec:strukura_laravel}). Jsou zde tyto rozdíly:
	
	\begin{itemize}
		\item Složka \textit{src} je nahrazena složkou \textit{js} ve složce \textit{resources}
		\item Složka \textit{public} je umístěna v kořenovém adresáři Laravel projektu a je sloučena se složkou \textit{dist}
		\item Složka \textit{node\_modules} společně se souborem \textit{package.json} je také umístěna v kořenovém adresáři Laravel projektu
	\end{itemize}
	
	Celkovou administrativu nad celým Vue projektem poté zajišťuje modul Laravel Mix podle konfiguračního souboru webpack.mix.js. Ten je také umístěn v kořenovém adresáři projektu. *cite https://laravel-mix.com/docs/6.0/vue* Díky celému tomuto přístupu může backend i frontend běžet jako jedna aplikace na jedné doméně.
	
	Jak již bylo zmíněno, zdrojový kód, tedy nejdůležitější část frontendové aplikace, leží standardně ve složce \textit{src} - v tomto projektu ve zmíněné složce \textit{js}. Oproti běžné samostatné Vue aplikaci je zde (ve složce \textit{js}) největší rozdíl v pojmenování souborů - soubor \textit{main.js} byl přejmenován pro potřeby Laravelu na \textit{app.js}. Také se navíc v této složce nachází Laravelem vygenerovaný soubor bootstrap.js pro další konfiguraci při kompilaci. Nakonec byla kvůli nevyužití odebrána složka \textit{assets} a pro další účely přidána složka \textit{apis} (viz obrázek \ref{fig:zdroj_kod_vue_rozdily}).
	
	\begin{figure}[h]
		\centering
		\subfloat[\centering]{{\includegraphics[height=5cm]{img/zdroj_kod_vue/src.png} }}
		\qquad
		\subfloat[\centering]{{\includegraphics[height=5cm]{img/zdroj_kod_vue/js.png} }}
		\caption{Porovnání obsahu složek \textit{src} (a) a \textit{js} (b)}
		\label{fig:zdroj_kod_vue_rozdily}
	\end{figure}

	Každý soubor a složka ve zmíněném adresáři \textit{js} má určitý význam. Ty jsou představeny níže (soubor bootstrap.js byl již popsán výše). Další rozbor jednotlivých souborů ve zmíněných složkách je obsažen v dalších sekcích této práce.
	\begin{itemize}
		\item Složka \textit{apis} obsahuje prostředky pro komunikaci s API (tedy s backendem)
		\item Složka \textit{components} obsahuje všechny komponenty, které jsou využívány v celé aplikaci
		\item Složka \textit{router} obsahuje prostředky pro přesměrovávání a definování cest ve frontendové aplikaci
		\item Složka \textit{store} obsahuje nástroje pro držení dat ve webovém prohlížeči
		\item Složka \textit{views} obsahuje veškeré pohledy, na které se můžeme pomocí definované cesty dostat
		\item Soubor \textit{app.js} je kořenový soubor aplikace shromažďující všechny reference na ostatní soubory (zabaluje celou aplikaci)
		\item Soubor \textit{App.vue} je kořenová šablona, která přijímá pohledy na základě cesty, které poté vykresluje (tvoří aplikační kostru HTML)
	\end{itemize}

		\subsubsection{Souhrn využitých balíčků}\label{sec:fe_packages}
		Pro přehlednost je zde vytvořena kapitola shrnující vybrané externí balíčky, které byly do frontendové části přidány, a jejich základní účel a funkcionalita.
		
		\begin{itemize}
			\item Balíček \uv{@braid/vue-formulate} (dále jako VueFormulate) je použit pro veškeré formulářové prvky. Kromě zprostředkování hotových komponent pro jednotlivé elementy formuláře také nabízí např. validaci dat.
			\item Balíček \uv{vue-chartjs} (zastřešující populární balíček \uv{chart.js}) je použit pro grafické vyobrazení dat ve formě grafů.
			\item Balíček \uv{randomcolor} je použit pro generování náhodných barev (využito u grafů k dynamickému rozlišení dat)
			\item Balíček \uv{uuid} je použit pro generování unikátního identifikátoru (především funkce \textit{uuidv4}, např. u vytváření formuláře k označování vytvořených otázek)
			\item Balíček \uv{vue-js-modal} je použit k vytváření a správě modálních oken.
			\item Balíček \uv{vue-toasted} je použit pro zobrazování zpráv o průběhu různých činností (např. zobrazení chybových hlášek).
		\end{itemize}
	
		Dále frontend obsahuje reference na automaticky přidané balíčky, které nejsou předmětem této práce. Lze z nich ale vyjmout např. \uv{vuex} a \uv{vuex-persist} (správa držení dat v prohlížeči), \uv{vue-router} (správa cest mezi jednotlivými částmi aplikace - směrovač) či \uv{axios} (komunikace s API).
	
	\subsection{Komunikace s Laravel API}
	Komunikace mezi frontendovou aplikací a Laravel API je pro celkovou funkčnost projektu klíčová - bez ní by nedávala existence této části žádný smysl. K těmto účelům je do projektu přidán balíček \uv{axios} - ten se stará o veškerou komunikaci mezi klientskou a serverovou stranou tzn. data dokáže odesílat i přijímat. Komunikace probíhá tak, že nějaký frontendový prvek (obvykle pohled) volá na metody zastřešené tímto balíčkem a na základě odpovědi ze serveru dostává data.
	
	Všechny tyto metody a soubory spojené s komunikací se nachází ve složce \textit{apis}. Zde je klíčový soubor \textit{Api.js}, ve kterém nejprve inicializujeme objekt, který celou komunikaci zastřešuje (ten pochází ze zmíněného balíčku) - tomu je vhodné přiřadit výchozí cestu, ze které bude později tvořit absolutní cesty, na které bude posílat specifické požadavky (v případě této implementace název domény + řetězec \textit{/api/}). Poté mu deklarujeme, co má provést v případě příchozí odpovědi ze serveru bez chyby a s chybou. V případě úspěšné akce (tedy navrácení dat bez chyby - obvykle s kódem 200) jsou data bez jakýchkoli úprav předána komponentě, která si je žádala. V opačném případě (tedy při chybě) je kromě vrácení obsahu příchozí zprávy také do konzole vypsána příslušná chyba. Vrácení příchozích dat s chybou tímto způsobem je určeno pro vytvoření chybového oznámení pro uživatele - aby uživatel věděl, proč daná operace selhala. Zároveň to slouží např. samotným pohledům - na základě chybové hlášky nebo chybového stavového kódu přizpůsobí své chování. Výpis do konzole sloužil primárně k vývoji aplikace - ničemu ale nevadil, proto byl v aplikaci ponechán. 
	
	Speciálním případem je situace, kdy se vyskytne chyba se stavovým kódem 401 nebo 419 (uživatel pravděpodobně chce přistoupit k obsahu, ke kterému nemá v současnou chvíli oprávnění např. protože mu vypršela relace přihlášení). V takovém případě se aplikace pokusí uživateli zneplatnit přihlášení a odstranit o něm záznam z prohlížeče a uživatele automaticky přesměrovat na pohled pro přihlášení.
	
	Další soubory ve zmíněném adresáři jsou \textit{Form.js} a \textit{User.js}. Ty obsahují konkrétní funkce s referencí na objekt deklarovaný v souboru \textit{Api.js}, přednastavenou metodu komunikace (ta musí korelovat s metodami na API viz sekce \ref{sec:routes}) a relativní cestu, na kterou budou požadavek směrovat (ta je spojena s předem definovanou výchozí cestou). Funkci je samozřejmě možné předat parametr (obvykle set dat), který je poté předán na server. Jak už z názvů souborů vyplývá - \textit{Form.js} zaštiťuje komunikaci týkající se manipulace s formuláři nebo s odpověďmi na formulář a \textit{User.js} zajišťuje komunikaci v rámci veškeré administrativy nad uživateli.
	
	\subsection{Komponenty}
	Komponenty jsou v podstatě \uv{díly k složení celé aplikace}. Díky nim nemusíme např. stránku pro vytvoření formuláře psát do jednoho nepřehledného a obrovského souboru, ale můžeme tento kód rozdělit. Výhodou také je, že když budeme tento komponentový styl zápisu kódu dodržovat, lze jeho části v jiných místech opětovně využít - předcházíme tím redundantnímu kódu, který zbytečně znepřehledňuje celý projekt. Všechny zmíněné komponenty jsou zapsány v souborech s koncovkou \textit{.vue} a disponují stejným jménem jako samotný soubor.
		
		\subsubsection{Základní komponenty} %Header, Loading, teor. App.vue
		Mezi základní komponenty jsou řazeny především ty, jejichž použití se opakuje v aplikaci nejčastěji - tedy \textit{Header} a \textit{Loading}. 
		
		Komponenta \textit{Header} slouží k vykreslení horního navigačního panelu s nabídkou dalších funkcionalit, který na základě přihlášení uživatele mění svůj obsah. Pokud není nikdo přihlášen, nabídka nabízí odkazy na přihlášení či registraci. Pokud se uživatel přihlásí, může se odsud dostat na stránku svého profilu, na vytvoření nového formuláře nebo se odhlásit ze svého účtu.
		
		Komponenta \textit{Loading} obsahuje jednoduchou animaci, která symbolizuje průběh načítání. Je využita v částech komponent a pohledů, kde je očekáván např. příjem nějakých dat z backendu, na který se musí čekat. Uživateli má vyobrazovat, že v aplikaci probíhá nějaká činnost.
		
		Dále zde lze zmínit i komponentu \textit{App} z kořenového adresáře frontendu, která již byla popsána v sekci \ref{sec:obecna_str_vuejs}. Lze dodat, že se v ní kromě předaných pohledů nachází i reference na komponentu \textit{Header} - ta musí být z podstaty věci vykreslena na všech cestách.
		
		\subsubsection{Komponenty elementů formuláře} %Elements, FormElement...*
		Tyto komponenty představují jednotlivé otázky a k ním příslušný vstup definovaný typem (pokud se nejedná o element nové stránky). Existují zde interaktivní komponenty elementů (použity např. ve vyobrazení celého formuláře k vyplnění) nebo komponenty statické (použity např. ve vyobrazení jednotlivých otázek při vytváření formuláře).
		
		Interaktivní komponenty elementů formuláře nalezneme v podsložce \textit{Elements} - zde se nachází tyto komponenty: \textit{BooleanInput}, \textit{DateInput}, \textit{NumberInput}, \textit{SelectInput} a \textit{TextInput} (vycházející z typů ze sekce \ref{sec:form_inputs_types}). Všechny tyto komponenty jsou si navzájem velmi podobné. Všechny využívají komponenty z balíčku \uv{VueFormulate}: \textit{FormulateInput} pro vykreslení samotné otázky a vytvoření vstupního pole podle datového typu pro odpověď (např. pro text je vygenerována oblast, do které lze vkládat textový řetězec) a \textit{FormulateErrors} pro výpis validační chyby (např. pokud má mít vkládaný text délku maximálně deset znaku a do oblasti vstupu je vložen text delší, tak se objeví chybové hlášení pod otázkou). V logické části jsou si taktéž všechny komponenty podobné - při jejich inicializaci se na základě přijmutích dat nastaví validační pravidla pro vstup a ID otázky, ke které samotná komponenta patří.
		
		Tyto interaktivní komponenty jsou poté zabaleny do dalších komponent. Komponenta \textit{FormElement} na základě příchozích dat (tj. typ otázky) vrací rodičovské komponentě \textit{FormElementsComponent} příslušný element formuláře rozebraný v předchozím odstavci. Zmíněná rodičovská komponenta poté figuruje především při vykreslování celého formuláře - má za úkol vykreslovat veškeré otázky formuláře s funkcí stránkování. Při inicializaci této komponenty se otázky nejprve rozdělí do příslušných polí, které představují stránky, na základě výskytu elementu nové stránky (tzn. pokud se při procházení všech elementů formuláře najde element nové stránky, tak je vytvořeno nové pole, do kterého se přesouvají další otázky). Po vzniku těchto polí představující stránky dojde k jejich vykreslení pomocí zmíněné komponenty \textit{FormElement}. Stránkování je zde zajištěno pomocí skrývání a odkrývání jednotlivých stránek pomocí CSS stylů ovládaných příslušnými tlačítky - takto to je řešeno kvůli udržení dat při přesunu na další stránku (nutné opatření kvůli zmíněnému balíčku k tvoření formulářových prvků).
		
		V rámci statických komponent zde existuje komponenta \textit{FormElementControl}. Ta slouží k vykreslování reprezentace otázek nebo elementu nové stránky v pohledech vytváření nebo úpravě formuláře. Jedná se o komponentu, která na základě předaného typu elementu (tj. otázka s příslušným typem nebo nová stránka) vrací příslušnou reprezentaci elementu (např. v případě textového vstupu vyobrazení textového pole, do kterého nelze vyplňovat, a samotné otázky). Při kliknutí na element otázky se objeví modální okno pro úpravu vybrané otázky - toto modální okno je popsáno dále v práci. U elementu nové stránky toto provést nelze, proto je zde připraven jen prokliknutelný text \uv{Remove}, který vysílá specifickou událost do rodičovské komponenty, která ho používá (slouží k odebrání elementu).
	
		\subsubsection{Modální okna} %Modals
		Zmíněné komponenty představují předpis vykreslení pro modální okna spravovaná balíčkem \uv{vue-js-modal}. Na jejich funkčnosti závisí několik pohledů. Tyto komponenty se nachází v podsložce \textit{Modals} - zde existují ještě další tři podsložky pojmenované podle oblasti, ve které jsou používány. Ty poté obsahují příslušné soubory komponent. Jedná se tedy o tyto složky: \textit{CreateForm}, \textit{FormResults}, \textit{FormReview}. 
		
			\subsubsubsection{Manipulace s elementy formuláře}
			První adresář \textit{CreateForm} obsahuje příslušné komponenty modálních oken a podkomponenty využívané především v pohledu vytváření formuláře. Později byly tyto komponenty využity i v pohledu pro úpravu formuláře - název tohoto adresáře se ale neměnil. Adresář obsahuje komponenty modálních oken \textit{ItemModal} a \textit{SelectChoiceModal}.
			
			Komponenta \textit{ItemModal} slouží k samotné tvorbě nového elementu nebo úpravě stávajícího elementu otázky ve formuláři - to je možné pomocí manipulace s dočasným úložištěm dat \textit{createFormStore} (viz \ref{sec:data_prohlizec_form}), se kterým se pracuje např. v rodičovském pohledu pro vytváření formuláře. K manipulaci s daty využívá příslušné metody inicializovaného úložiště.
			
			Celé modální okno je zobrazeno na základě vyžádání rodičovského prvku (např. pohledu pro vytvoření formuláře) a je tvořeno zejména komponentou \textit{FormulateForm} (z balíčku \uv{VueFormulate}), ve které jsou vnořeny všechny ovládací prvky (tvořené komponentami \textit{FormulateInput} a \textit{FormulateErrors}) - ty shromažďují veškeré informace o elementu otázky - tedy typ odpovědi na otázku (např. text nebo číslo), samotnou otázku, zda je otázka povinná a další validační pravidla, která jsou vázána na typ otázky (tato pravidla vychází ze sekce \ref{sec:form_inputs_types}). Ovládací prvky dalších validačních pravidel jsou další samotné komponenty připojované do modálního okna - každý typ otázky má pro tyto účely svou podkomponentu (jedná se o \textit{DateItem}, \textit{NumberItem}, \textit{SelectItem} a \textit{TextItem} ve stejné složce). Tyto prvku mohou být už předvyplněny v situaci, kdy je upravován existující element (komponenta může přijímat vstupní hodnoty). Data jsou poté na základě stisknutí příslušného tlačítka (např. s účelem vytvoření nového elementu s textem \uv{Create}) z modálního okna předána příslušné metodě úložiště, která si data sama zpracuje.
			
			Známou komplikací je samozřejmě zpracování otázek s možnostmi odpovědi. K těmto účelům je v komponentě \textit{SelectItem} vnořena další podkomponenta \textit{SelectChoicesComponent}. Tato komponenta poskytuje administrativu nad možnostmi odpovědi - k tomu používá vlastní dočasné úložiště \textit{createFormChoicesStore}, které má stejné metody jako již zmíněné úložiště \textit{createFormStore} (data z celé této komponenty samozřejmě podléhají i manipulaci s daty rodičovské komponentě \textit{ItemModal}). V komponentě jsou kromě tlačítka pro přidání možnosti odpovědi vykreslovány možnosti odpovědi. Ty lze samozřejmě modifikovat - při kliknutí na ně se zobrazí další modální okno \textit{SelectChoiceModal}, ve kterém ji lze upravit příslušné hodnoty (toto okno je používáno i k přidávání těchto možností). Toto okno poté data předává zmíněnému poskytnutému úložišti. Zajímavou funkcí, kterou lze u \textit{SelectChoicesComponent} zmínit, je hlídání hodnoty \textit{hasHiddenLabel} pocházející z rodičovské komponenty (ta značí, zda možnosti odpovědi mají skrytý popisek - tj. škála) - pokud je tato hodnota změněna z \textit{true} na \textit{false}, tak jsou automaticky tyto skryté popisky odstraněny. V opačném případě komponenta automaticky očísluje možnosti odpovědi čísly 1,2,3,...
			
			\subsubsubsection{Výsledky formuláře}
			Dalším adresářem je \textit{FormResults} obsahující pouze jedno modální okno. To je využito zejména v pohledu pro zobrazení výsledků formuláře. Toto okno je znovu tvořeno především prvky z balíčku \uv{VueFormulate}. 
			
			Před inicializací přijímá všechny otázky formuláře, hodnotu vyjadřující zda je formulář už zveřejněn a identifikátor formuláře. Poté, na základě přijatých dat, vypisuje jeho současný stav a možnosti, co lze s formulářem provést. Klíčová je hodnota vyjadřující zda jsou už výsledky veřejné - pokud se rovná \textit{false}, je zobrazeno jen prázdné zaškrtávací políčko vyjadřující, že výsledky zveřejněny nejsou. Pokud se hodnota rovná \textit{true}, jsou vypsány i všechny otázky, u kterých lze označit, která z nich bude veřejně viditelná. Zmíněnou hodnotu můžeme pomocí zaškrtávacího políčka samozřejmě měnit - s tím se i dynamicky změní rozložení okna. Poté, co navolíme příslušné parametry a klikneme na tlačítko pro uložení nastavení, je na server poslán požadavek s příslušnými údaji. Ten je zpracován metodou \textit{publishResults}, která je popsána v sekci \ref{sec:form_publish_results}. Po úspěšném provedení činnosti se okno zavírá a celá stránka je přenačtena.
			
			\subsubsubsection{Manipulace s celým formulářem}\label{sec:modalni_okna_man_form}
			Posledním adresářem je \textit{FormReview}. Ten obsahuje dvě modální okna použitá v pohledu pro zobrazení souhrnu informací o formuláři. Jedná se o komponenty \textit{FormAccessibilityModal} a \textit{FormDuplicationModal}.
			
			Komponenta \textit{FormAccessibilityModal} slouží k úpravě přístupnosti formuláře. Je také tvořena především komponentami z balíčku \uv{VueFormulate} a ke svému správnému chodu potřebuje při vytvoření nejprve dostat hodnotu vyjadřující současnou přístupnost (zda je formulář veřejný nebo privátní) a identifikátor formuláře - pokud se jedná o privátní formulář, musí být předány i emaily, na které byla už zaslána pozvánka. Na základě těchto dat je vykresleno příslušné rozložení modálního okna - toto rozložení lze měnit na základě změny hodnoty vyjadřující přístupnost. Pokud je zmíněná hodnota \textit{false}, zobrazí se jen související prázdné zaškrtávací políčko. Pokud se ale rovná \textit{true}, zobrazují se zde i další dva ovládací prvky. V prvním lze vidět emaily, na které již byla zaslána pozvánka - ty lze pro zneplatnění pozvánky označit. V druhém můžeme zase přidat emaily, na které bude poslána nová pozvánka k vyplnění formuláře. Nakonec (po kliknutí na tlačítko vyjadřující provedení změny) se příslušná data pošlou na server. Ta jsou zpracována metodou \textit{updateAccess}, která je popsána v sekci \ref{sec:form_accessibility}. Modální okno se poté zavře a stránka je přenačtena.
			
			Druhá komponenta \textit{FormDuplicationModal} nám umožňuje duplikovat vybraný formulář. I toto modální okno je tvořeno komponentami z balíčku \uv{VueFormulate} a při inicializaci očekává základní data o formuláři - tedy jeho název, popisek, datum zveřejnění, datum ukončení zveřejnění a zda se jedná o formulář veřejný či privátní (u privátního také emaily, které dostali pozvánku). Tato data jsou dočasně uložena v úložišti \textit{duplicateFormStore}, které je popsáno v sekci \ref{sec:data_prohlizec_form}. Tyto údaje lze libovolně měnit - pokud jsou zadány neplatné hodnoty (např. datum zveřejnění je dále v čase než datum ukončení zveřejnění), tak nás okno bude informovat příslušným upozorněním. Poté, co klikneme na příslušné tlačítko pro duplikaci formuláře, je odeslán na server požadavek, který je vyřízen metodou \textit{duplicateWithAuth}, jež je popsána v sekci \ref{sec:form_dupl}. Po tomto úkonu je modální okno zavřeno a stránka přenačtena.
		
		\subsubsection{Komponenty pro zobrazení výsledků} %ResultComponents
		Dalšími komponentami, které jsou zde popsány, jsou komponenty pro grafické znázornění dat v pohledu pro zobrazení výsledků formuláře (pro vyobrazení dat ve formě grafů zde byl použit balíček \uv{vue-chartjs}). Nachází se v adresáři \textit{ResultComponents}, který obsahuje komponenty \textit{DetailedResultsTable}, \textit{ResultComponent} a \textit{ResultRender}. Také se zde nachází podsložka \textit{Views}, ve které lze najít předpřipravené podkomponenty pro konkrétní styl vykreslení dat - \textit{Bar} pro vykreslování sloupcových grafů, \textit{List} pro vykreslování jednoduché tabulky stylem klíč/hodnota (resp. ID odpovědi/odpověď) a \textit{Pie} pro vykreslování koláčových grafů. Tyto podkomponenty jsou poté předávány dál pomocí zmíněné komponenty \textit{ResultRender}, která je vrací na základě předaného typu a dat (tzn. výsledky z otázek s možnostmi odpovědi nebo z otázek Ano/Ne jsou zobrazovány jako grafy - lze si mezi zmíněnými styly grafu po načtení vybrat, ostatní jako zmíněná tabulka).
		
		Popsaná komponenta \textit{ResultRender} je poté využita ve zmíněné komponentě \textit{ResultComponents}. Pro správnou funkcionalitu komponenta \textit{ResultComponents} musí pro získání příslušného vyobrazení dat samozřejmě předaná data zpracovat do validního formátu a poté je poskytnout komponentě \textit{ResultRender}. Nakonec zmíněná rodičovská komponenta \textit{ResultComponents} přidá k vyobrazení dat (např. ve formě grafu) samotnou otázku a celý obsah je předán do pohledu pro zobrazení výsledků.
		
		Nelze opomenout komponentu \textit{DetailedResultsTable}, která slouží jako alternativní zobrazení všech výsledků formuláře. (ve výchozím nastavení se zobrazují data ve zmíněných grafech nebo v tabulce pro každou otázku). V podstatě data zobrazuje v podobném formátu, jako jsou vysázeny při exportu dat (tzn. v tabulce viz sekce \ref{sec:form_comp_export}). Oproti exportu se vyobrazení dat liší v tom, že v buňkách, kde není žádná hodnota (\textit{null}), tam je umístěn znak pomlčky (\uv{-}).
		
		\subsubsection{Komponenta karty formuláře}
		Jednoduchá komponenta, kterou nebylo možné zařadit mezi ostatní výše, s názvem \textit{FormCard} slouží jako předpis pro vykreslování karty formuláře. Ta je použita v pohledu úvodní stránky v případě, že je uživatel přihlášen a že už nějaké formuláře vlastní. Po kliknutí na kartu se přihlášený uživatel dostane na pohled se zobrazením souhrnu informací o požadovaném formuláři. Sama o sobě karta zobrazuje název a popisek formuláře či data spuštění a ukončení zveřejnění formuláře.
	
	\subsection{Směrovač}\label{sec:fe_router} %%Router
	K tomu, abychom mohli přecházet mezi částmi aplikace, je nutné, aby v aplikaci existoval nějaký komplexní systém, který by nad tímto problémem měl kontrolu. To zde řeší tzv. směrovač (běžně známý pod anglickým názvem Router), který je importován z balíčku \uv{vue-router}. Tento směrovač má za úkol na základě definovaných možných cest v aplikaci vracet příslušný pohled uživateli.
	
	Směrovač implementovaný v tomto projektu je umístěn ve složce \textit{router} v souboru \textit{index.js}. V tomto souboru jsou nejprve vydefinovány všechny relativní cesty, ke kterým je přiřazen hlavně příslušný pohled, který je při vstupu na nějakou z cest vrácen uživateli - některé mohou disponovat i metadaty (např. zda cesta vyžaduje pro vykreslení přihlášení uživatele). Poté je inicializován objekt tohoto směrovače, kterému jsou předány především data jako výchozí cesta (v této implementaci název domény), ke které jsou relativní cesty ve výsledku připojeny, a samotná kolekce vydefinovaných cest. 
	
	Nakonec lze určit i specifické chování pro zmíněná metadata u cest, což je zde také využito a popsáno níže:
	\begin{itemize}
		\item Pokud cesta vyžaduje přístup bez přihlášení a uživatel je přihlášen, je přesměrován na domovskou stránku (např. pokus o přístup na přihlašovací stránku). Pokud není přihlášen, je přesměrován na stránku na zvolené cestě.
		\item Pokud cesta vyžaduje přístup s přihlášením a uživatel je přihlášen, je přesměrován na stránku na zvolené cestě. Pokud není, je přesměrován na pohled pro přihlášení (např. pokus o přístup na pohled zobrazující souhrn informací o formuláři, který je určen jen přihlášenému majiteli formuláře).
		\item Pokud nemá cesta žádná známá metadata, je uživatel přesměrován na stránku na zvolené cestě.
	\end{itemize}

	\subsection{Držení dat v prohlížeči} %% Store
	Možnost držení dat v prohlížeči je další ze základních atributů pro správnou funkčnost aplikace. Ať už se jedná o soubory cookies nebo o dočasné úložiště v paměti alokované přímo Javascriptem - jednoduše je to potřeba. V tomto projektu byly využity oba zmíněné přístupy.
	
	Soubory spojené s držením dat pomocí Javascriptu (označovány jako Stores) jsou umístěné ve složce \textit{store} - jedná se o soubor \textit{index.js} a soubor \textit{MyStoreClass.js}.
	
		\subsubsection{Přihlášení uživatele}
		V rámci přihlášení uživatele lze držení dat rozdělit na dvě větve: pomocí cookies a pomocí držení dat v Javascriptu za využití balíčku \uv{vuex}. 
		
		V cookies jsou drženy šifrované tokeny, které jsou spravované přímo backendovým balíčkem pro zabezpečení single-page aplikací \uv{Laravel Sanctum} a slouží k ověřování přihlášení (jsou autonomně posílány společně s každým požadavkem na server - ten na základě jejich platnosti provede požadovaný úkon). 
		
		Naproti tomu přihlášení drženo pomocí \uv{vuex} je zde primárně pro funkčnost metadat u směrovače (viz sekce \ref{sec:fe_router}) - na základě tohoto drženého stavu může přihlášený uživatel přistupovat na stránky, které jsou přístupné pouze s přihlášením. Toto je definováno v souboru \textit{index.js}, kde probíhá inicializace objektu z balíčku \uv{vuex}, který má deklarován příslušné proměnné držící stav (states), metody pro vrácení hodnot proměnných držících stav (getters), metody pro cílenou změnu hodnot (mutations) a další procedury ve formě metod (actions - např. procedura pro přihlášení uživatele). K tomu, aby data nebyla smazána při přenačtení stránky, musí být přidán zmíněnému objektu plugin \textit{VuexPersistence} z balíčku \uv{vuex-persist}, který toto chování zprostředkovává.
		
		\subsubsection{Data pro manipulaci s formuláři}\label{sec:data_prohlizec_form}
		K tomu, abychom mohli vytvářet, upravovat nebo duplikovat formuláře, bylo nutné jejich data dočasně (do odeslání požadavku na server) držet v prohlížeči. K těmto účelům vznikla tři dočasná \uv{skladiště} dat - \textit{createFormStore}, \textit{createFormChoicesStore} a \textit{duplicateFormStore}.
		
		První dva zmíněné objekty (tedy \textit{createFormStore} a \textit{createFormChoicesStore}) jsou určené pro vytváření a úpravu formuláře a jsou ve své podstatě naprosto identické, jelikož se jedná o instance pro projekt vytvořené třídy \textit{MyStore}, která se nachází v souboru \textit{MyStoreClass.js}. Třída \textit{MyStore} obsahuje proměnnou pro ukládání dat, což je objekt, ve kterém je vytvořeno pole pro ukládání hodnot kolekce dat. Kromě této proměnné třída disponuje různými metodami: \textit{getItems} vrací všechny prvky kolekce, \textit{addItem} ukládá předaná data z parametru do kolekce, \textit{refreshItemsOrder} obnovuje hodnotu pořadí každému prvku v kolekci (tzv. postupně očísluje prvky), \textit{sortItemsByOrder} řadí prvky v kolekci podle hodnoty pořadí, \textit{changeItem} dokáže nahradit existující prvek v kolekci na základě shody ID, \textit{deleteItem} smaže prvek z kolekce na základě ID, \textit{clearStore} vymaže všechny prvky z kolekce a \textit{setItems} přepíše celou kolekci předanou kolekcí.
		
		Třetí objekt (\textit{duplicateFormStore}) slouží k držení dat pro duplikaci formuláře. Oproti zmíněným instancím se jedná o \uv{skladiště}, které je připraveno pro konkrétní data související s duplikací. V rámci proměnných, ve kterých dočasně drží data, jsou zde vlastně deklarovány pole pro příslušné ovládací prvky modálního okna pro duplikaci (viz sekce \ref{sec:modalni_okna_man_form}). Dále obsahuje standardní metody pro smazání, přepsání a vrácení uložených dat (\textit{clearData}, \textit{setData} a \textit{getData}), ale také metodu pro validaci těchto dat (\textit{validateData}, která na vrací případnou chybovou hlášku pro zobrazení v modálním okně) či metodu pro zjištění, zda je úložiště prázdné neboli ve výchozím stavu (\textit{isStoreEmpty}).
		
		Lze namítat, že pro ukládání těchto dočasných dat je možné využít lokální úložiště Vue komponenty \textit{data} - zde to ale možné nebylo, jelikož bylo potřeba předávat data mezi více komponentami. Zároveň, díky tomuto přístupu, jsem mohl s daty manipulovat v určitých situacích jednodušeji za pomocí předpřipravených metod.
	
	\subsection{Pohledy}
		
		\subsubsection{Přihlašování a profil}
		
		\subsubsection{Domovská stránka}
		
		\subsubsection{Formulář}
		
		\subsubsection{Tvorba formuláře}
		
		\subsubsection{Úprava formuláře}
		
		\subsubsection{Souhrn informací o formuláři}
		
		\subsubsection{Výsledky formuláře}
		
		\subsubsection{Ostatní pohledy}
		%% not found, about

\chapter{Závěr}
Vznikla ...

Lorem ipsum haf baf haf. Co lze zlepšit.




\begin{thebibliography}{50}

\bibitem{GoogleForms1}
Google Forms: Free Online Form Creator | Google Workspace. \textit{Google} [online]. [cit. 2022-01-15]. Dostupné z: https://www.google.com/forms/about/

\bibitem{GoogleForms2}
DEMAREST, Abigail Abesamis. What are Google Forms? Everything you need to know about Google Workspace's online form builder. \textit{Business Insider} [online]. 2021, 22.1.2021 [cit. 2022-01-15]. Dostupné z: https://www.businessinsider.com/what-is-google-forms

\bibitem{MSForms1}
Microsoft Forms | Průzkumy, hlasování a kvízy. \textit{Microsoft Forms} [online]. [cit. 2022-01-30]. Dostupné z: https://www.microsoft.com/cs-cz/microsoft-365/online-surveys-polls-quizzes

\bibitem{MSForms2}
Microsoft Forms --- Užitečný nástroj, díky němuž se mnoho dozvíte. \textit{Microsoft Forms} [online]. 2021, 3.2.2021 [cit. 2022-01-30]. Dostupné z: https://manica.cz/microsoft-forms-uzitecny-nastroj-diky-nemuz-se-mnoho-dozvite/

\bibitem{Survio}
Dotazník zdarma: Vytvořit online dotazník. \textit{Survio®} [online]. [cit. 2022-01-30]. Dostupné z: https://www.survio.com/cs/

\bibitem{FEvsBE}
VERNEROVÁ, Sára. Front end vs. Back end - jaký je mezi nimi rozdíl?. \textit{Apitree} [online]. 2021, 12.9.2021 [cit. 2022-02-04]. Dostupné z: https://www.apitree.cz/blog/front-end-vs-back-end-jaky-je-mezi-nimi-rozdil

\bibitem{JS1}
JavaScript. \textit{Wikipedia} [online]. 2021, 16.12.2021 [cit. 2022-02-04]. Dostupné z: https://cs.wikipedia.org/wiki/JavaScript

\bibitem{JS2}
KOĎOUSKOVÁ, Barbora. JavaScript pro začátečníky: co to je a jak funguje. \textit{Rascasone} [online]. 2022, 28.01.2022 [cit. 2022-01-23]. Dostupné z: https://www.rascasone.com/cs/blog/co-je-javascript-pro-zacatecniky

\bibitem{VueJS1}
Introduction. \textit{Vue.js} [online]. [cit. 2022-01-23]. Dostupné z: https://vuejs.org/v2/guide/\#What-is-Vue-js

\bibitem{VueJSSyntax}
SFC Syntax Specification. \textit{Vue.js} [online]. [cit. 2022-01-23]. Dostupné z: https://v3.vuejs.org/api/sfc-spec.html\#custom-blocks

\bibitem{VueJS2}
Úvod do frameworku Vue.js. \textit{Vue manuál} [online]. [cit. 2022-01-23]. Dostupné z: https://vue.baraja.cz/uvod-do-vue

\bibitem{Bootstrap1}
The most popular HTML, CSS, and JavaScript framework for developing responsive, mobile first projects on the web. \textit{GitHub} [online]. [cit. 2022-01-30]. Dostupné z: https://github.com/twbs/bootstrap

\bibitem{Bootstrap2}
Bootstrap (front-end framework). \textit{Wikipedia} [online]. 2022, 2022 [cit. 2022-01-30]. Dostupné z: https://en.wikipedia.org/wiki/Bootstrap\_(front-end\_framework)

\bibitem{Bootstrap3}
About · Bootstrap v5.1. \textit{Bootstrap} [online]. [cit. 2022-01-30]. Dostupné z: https://getbootstrap.com/docs/5.1/about/overview/

\bibitem{BE1}
ŠTRÁFELDA, Jan. Co je backend. \textit{Jan Štráfelda} [online]. [cit. 2022-01-30]. Dostupné z: https://www.strafelda.cz/backend

\bibitem{BE2}
Backend. \textit{Shoptet.cz} [online]. [cit. 2022-01-30]. Dostupné z: https://www.shoptet.cz/slovnik-pojmu/backend/

\bibitem{Laravel1}
Introduction. \textit{Laravel} [online]. [cit. 2022-01-30]. Dostupné z: https://laravel.com/docs/4.2/introduction

\bibitem{LaravelMVC}
How Laravel implements MVC and how to use it effectively. \textit{Pusher} [online]. 2018, 22.8.2018 [cit. 2022-01-31]. Dostupné z: https://blog.pusher.com/laravel-mvc-use/

\bibitem{LaravelEco}
Laravel Ecosystem Tools --- How Development Can be Faster and More Effective. \textit{ASPER BROTHERS} [online]. 2019, 20.4.2019 [cit. 2022-01-30]. Dostupné z: https://asperbrothers.com/blog/laravel-ecosystem-tools/

\bibitem{MVC}
ČÁPKA, David. MVC architektura. \textit{Itnetwork.cz} [online]. [cit. 2022-01-31]. Dostupné z: https://www.itnetwork.cz/navrh/mvc-architektura-navrhovy-vzor

\bibitem{Webhosting}
Webhosting. \textit{Wikipedia} [online]. 2021, 27.10.2021 [cit. 2022-01-30]. Dostupné z: https://cs.wikipedia.org/wiki/Webhosting

\bibitem{DO1}
SALEEM, Salman. What is DigitalOcean and why should you host apps on it?. \textit{The Official Cloudways Blog} [online]. 2021, 20.12.2021 [cit. 2022-01-30]. Dostupné z: https://www.cloudways.com/blog/what-is-digital-ocean/

\bibitem{DO2}
DigitalOcean Products | Designed for Developers, Built for Businesses. \textit{Digitalocean.com} [online]. [cit. 2022-01-30]. Dostupné z: https://www.digitalocean.com/products

\bibitem{DO3}
DigitalOcean App Platform: Build, Deploy and Scale Apps Quickly. \textit{Digitalocean.com} [online]. [cit. 2022-01-30]. Dostupné z: https://www.digitalocean.com/products/app-platform

\bibitem{Heroku1}
About Heroku. \textit{Heroku} [online]. [cit. 2022-01-30]. Dostupné z: https://www.heroku.com/about

\bibitem{Heroku2}
The Heroku product suite. \textit{Heroku} [online]. [cit. 2022-01-30]. Dostupné z: https://www.heroku.com/products

\bibitem{DBSummary}
What is a database?. \textit{Oracle | Cloud Applications and Cloud Platform} [online]. [cit. 2022-01-31]. Dostupné z: https://www.oracle.com/cz/database/what-is-database/

\bibitem{DBModel}
KUČEROVÁ, Helena. Databázový model. \textit{KTD: Česká terminologická databáze knihovnictví a informační vědy (TDKIV)} [online]. Praha: Národní knihovna ČR, 2003 [cit. 2022-02-04]. Dostupné z: \url{https://aleph.nkp.cz/F/?func=direct&doc\_number=000000091&local\_base=KTD}

\bibitem{HierarchDB}
KUČEROVÁ, Helena. Hierarchická databáze. \textit{KTD: Česká terminologická databáze knihovnictví a informační vědy (TDKIV)} [online]. Praha: Národní knihovna ČR, 2003 [cit. 2022-02-04]. Dostupné z: \url{https://aleph.nkp.cz/F/?func=direct&doc\_number=000000103&local\_base=KTD}

\bibitem{SitDB}
KUČEROVÁ, Helena. Síťová databáze. \textit{KTD: Česká terminologická databáze knihovnictví a informační vědy (TDKIV)} [online]. Praha: Národní knihovna ČR, 2003 [cit. 2022-02-04]. Dostupné z: \url{https://aleph.nkp.cz/F/?func=direct&doc\_number=000000128&local\_base=KTD}

\bibitem{RelacDB}
KUČEROVÁ, Helena. Relační databáze. \textit{KTD: Česká terminologická databáze knihovnictví a informační vědy (TDKIV)} [online]. Praha: Národní knihovna ČR, 2003 [cit. 2022-02-04]. Dostupné z: \url{https://aleph.nkp.cz/F/?func=direct&doc\_number=000000126&local\_base=KTD}

\bibitem{OOPDB}
MEHRA, Puran. What Are Object-Oriented Databases And Their Advantages. \textit{C\# Corner} [online]. 6.9.2019 [cit. 2022-02-04]. Dostupné z: https://www.c-sharpcorner.com/article/what-are-object-oriented-databases-and-their-advantages2/

\bibitem{DotazJazyk}
KUČEROVÁ, Helena. Dotazovací jazyk. \textit{KTD: Česká terminologická databáze knihovnictví a informační vědy (TDKIV)} [online]. Praha: Národní knihovna ČR, 2003 [cit. 2022-02-05]. Dostupné z: \url{https://aleph.nkp.cz/F/?func=direct&doc\_number=000000098&local\_base=KTD}

\bibitem{SQL}
BRITANNICA, The Editors of Encyclopaedia. SQL. \textit{Encyclopedia Britannica} [online]. 16.6.2021 [cit. 2022-02-05]. Dostupné z: https://www.britannica.com/technology/SQL

\bibitem{Transakce}
What is a Transaction? - Win32 apps. \textit{Microsoft} [online]. 2021, 1.7.2021 [cit. 2022-02-05]. Dostupné z: https://docs.microsoft.com/en-us/windows/win32/ktm/what-is-a-transaction

\bibitem{ACID}
IAN. What does ACID mean in Database Systems?. \textit{Database.Guide} [online]. 2016, 20.6.2016 [cit. 2022-02-05]. Dostupné z: https://database.guide/what-is-acid-in-databases/

\bibitem{PostgreSQL}
About. \textit{PostgreSQL} [online]. [cit. 2022-02-05]. Dostupné z: https://www.postgresql.org/about/


\end{thebibliography}

%% obrázky 
\listoffigures

%% tabulky
\listoftables

%\appendix %% začínají přílohy
%\titleformat{\chapter}[block]{\scshape\bfseries\LARGE}{Příloha \thechapter}{10pt}{\vspace{0pt}}[\vspace{-22pt}] %% nastavení nadpisu u příloh
%
%\chapter{%Příloha A 
%Spot diagramy a další }

\appendix
\chapter[Příloha 1: Schéma databáze]{Schéma databáze}
	%lze odebrat zobrazování jednotlivých příloh v obsahu
	%\addtocontents{toc}{\protect\setcounter{tocdepth}{0}}
		\begin{figure}[hbtp]
			\centering %% příkaz, který ti obrázek zarovná na střed
			\includegraphics[height=0.61\paperheight]{img/appendix/db_diagram.png} %% vložení samotného obrátku
			\label{fig:db_diagram} %% označení až budeš chtít na obrázek odkazovat
		\end{figure}
	%pro školu
	%\includepdf[pages={1-},fitpaper]{pdf/db_diagram_A3.pdf}	
	%\cleardoublepage
	%\includepdf[pages={1-}]{pdf/zadani.pdf}	

\end{document}